\documentclass{article}
%\documentclass[twocolumn]{article}
%\documentclass[onecolumn]{article}
% \usepackage{scrtime} % for \thistime (this package MUST be listed first!)
\DeclareUnicodeCharacter{0301}{\'{e}}
\usepackage{times}
\usepackage{graphicx}
\usepackage{float}
\usepackage[margin=0.75in]{geometry}
\usepackage{fancyhdr}
\usepackage{caption}
\usepackage{notoccite}
\usepackage{pgfplotstable}
%\usepackage[round]{natbib}
%\setcitestyle{aysep={}} %removes the comma between the author and year in citations
%\usepackage{underscore}
\usepackage{pdfpages}
\usepackage{xcolor,colortbl}%for changing cell colour
\usepackage[normalem]{ulem}
\useunder{\uline}{\ul}{}
\usepackage{xspace}
\usepackage{booktabs}
\usepackage{capt-of}
\pagestyle{fancy}
\setlength{\headheight}{15.2pt}
\setlength{\headsep}{13 pt}
\setlength{\parindent}{28 pt}
\setlength{\parskip}{12 pt}
\pagestyle{fancyplain}
\usepackage[T1]{fontenc}
\usepackage{amsmath}
% \usepackage{color,amsmath,amssymb,amsthm,mathrsfs,amsfonts,dsfont}
\usepackage{xspace}
\usepackage{tikz-cd}
\usepackage{tikz}
\usetikzlibrary{decorations.markings}
\usetikzlibrary{calc, arrows}
\usetikzlibrary{external}
\usepackage{pgfplots}
\pgfplotsset{layers/my layer set/.define layer set={background,main,foreground}{},
        set layers=my layer set,}

\usepackage{listings}
\usepackage{xcolor}

\definecolor{codegreen}{rgb}{0,0.6,0}
\definecolor{codegray}{rgb}{0.5,0.5,0.5}
\definecolor{codepurple}{rgb}{0.58,0,0.82}
\definecolor{backcolour}{rgb}{0.95,0.95,0.92}

%for code
\lstdefinestyle{mystyle}{
	backgroundcolor=\color{backcolour},   
	commentstyle=\color{codegreen},
	keywordstyle=\color{magenta},
	numberstyle=\tiny\color{codegray},
	stringstyle=\color{codepurple},
	basicstyle=\ttfamily\footnotesize,
	breakatwhitespace=false,         
	breaklines=true,                 
	captionpos=b,                    
	keepspaces=true,                 
	numbers=left,                    
	numbersep=5pt,                  
	showspaces=false,                
	showstringspaces=false,
	showtabs=false,                  
	tabsize=2
}

\lstset{style=mystyle}
% \usetikzlibrary{pgfplots.clickable}
% \usepgfplotslibrary{clickable}
% tables
\usepackage{longtable}
\usepackage{booktabs}
\usepackage{multicol}
\usepackage{multirow}
% figs
%\usepackage{subfig}% http://ctan.org/pkg/subfig
\usepackage{subcaption}
%\newsubfloat{figure}% Allow sub-figures
\usepackage{caption}%lable fig caption as fig
%\captionsetup[subfigure]{labelfont=bf, justification=raggedright, labelformat=empty} %no caption label
\usepackage{stackengine} %places caption inside figure?
\captionsetup{subrefformat=parens} %when you reference the subcaption it will be (a) for example %{labelfont={color=blue}}
%\captionsetup[subfigure]{labelsep=colon}


% \usepackage{acronym}
% \usepackage{lineno}%for line numbers
%%%%%%%%%%%%%%%%%%%%%%%%%%%%%%%%%%%%%%%%%%%%%%%%%%%%%%%%%%%%%%%%%%%%%%%%%%%%%%%%
% BIBLIOGRAPHY
%%%%%%%%%%%%%%%%%%%%%%%%%%%%%%%%%%%%%%%%%%%%%%%%%%%%%%%%%%%%%%%%%%%%%%%%%%%%%%%%
\usepackage[backend=biber, giveninits=true, doi=false, isbn=false, natbib=true, url=true, eprint=false, style=authoryear-comp, sorting=nyt, sortcites=ynt, maxcitenames=2, maxbibnames=10, minbibnames = 10, uniquename=false, uniquelist=false, dashed=false]{biblatex} % can change the maxbibnames to cut long author lists to specified length followed by et al., currently set to 99.

%% bibliography for each chapter...
\DeclareFieldFormat[article,inbook,incollection,inproceedings,patent,thesis,unpublished]{title}{#1\isdot} % removes quotes around title
\renewbibmacro*{volume+number+eid}{%
	\printfield{volume}%
	%  \setunit*{\adddot}% DELETED
	\printfield{number}%
	\setunit{\space}%
	\printfield{eid}}
\DeclareFieldFormat[article]{number}{\mkbibparens{#1}}
%\renewcommand*{\newunitpunct}{\space} % remove period after date, but I like it. 
\renewbibmacro{in:}{\ifentrytype{article}{}{\printtext{\bibstring{in}\intitlepunct}}} % this remove the "In: Journal Name" from articles in the bibliography, which happens with the ynt 
\renewbibmacro*{note+pages}{%
	\printfield{note}%
	\setunit{,\space}% could add punctuation here for after volume
	\printfield{pages}%
	\newunit}    
\DefineBibliographyStrings{english}{% clears the pp from pages
	page = {\ifbibliography{}{\adddot}},
	pages = {\ifbibliography{}{\adddot}},
} 
\DeclareFieldFormat{journaltitle}{#1\isdot}
\renewcommand*{\revsdnamepunct}{}%remove comma between last name and first name
\DeclareNameAlias{sortname}{family-given}
% \DeclareNameAlias{sortname}{last-first}
\renewcommand*{\nameyeardelim}{\addspace} % remove comma in text between name and date
\addbibresource{ABC1.bib} % The filename of the bibliography
\usepackage[autostyle=true]{csquotes} % Required to generate language-dependent quotes in the bibliography
\renewrobustcmd*{\bibinitperiod}{}
% you'll have to play with the citation styles to resemble the standard in your field, or just leave them as is here. 
% or, if there is a bst file you like, just get rid of all this biblatex stuff and go back to bibtex. 
%%%%%%%%%%%%%%%%%%%%%%%%%%%%%%%%%%%%%%%%%%%%%%%%%%%%%%%%%%%%%%%%%%%%%%%%%%%%%%%%
%
% generally hyperref needs to be loaded last
\usepackage[hidelinks,colorlinks=true,linkcolor=blue,citecolor=blue,urlcolor=darkbrown]{hyperref}
%\usepackage[hidelinks,colorlinks=false,citecolor=blue,urlcolor=darkbrown]{hyperref}
\tikzexternalize

\lhead{Cells at War: Game-Based Learning} %This needs to change
\rhead{Bianca Flaim, Alexander Turco}
\title{\sc Cells at War: The Playfulness of Game-Based Learning}
\author{\sc Bianca Flaim, Alexander Turco and Rosa da Silva}

\providecommand{\figref}[1]{(Figure \ref{#1})}  %what?
\providecommand{\tabref}[1]{(Table \ref{#1})}  %what?
\providecommand{\e}[1]{\ensuremath{\times 10^{#1}}}
\newcommand{\seg}{\texttt{Seg}\xspace}
\newcommand{\ecoli}{\mbox{\textit{E.\,coli}}\xspace}
\newcommand{\sclong}{\textit{Saccharomyces cerevisiae}\xspace}
\newcommand{\scshrt}{\mbox{\textit{S.\,cerevisiae}}\xspace}
\newcommand{\sce}{\mbox{\textit{S.\,cerevisiae}}\xspace}
\newcommand{\hslong}{\textit{Homo sapiens}\xspace}
\newcommand{\hsshrt}{\mbox{\textit{H.\,sapiens}}\xspace}
\newcommand{\hse}{\mbox{\textit{H.\,sapiens}}\xspace}
\newcommand{\celong}{\textit{Caenorhabditis elegans}\xspace}
\newcommand{\ceshrt}{\mbox{\textit{C.\,elegans}}\xspace}
\newcommand{\dmlong}{\textit{Drosophila melanogaster}\xspace}
\newcommand{\dmshrt}{\mbox{\textit{D.\,melanogaster}}\xspace}
\newcommand{\atlong}{\textit{Arabidopsis thaliana}\xspace}
\newcommand{\atshrt}{\mbox{\textit{A.\,thaliana}}\xspace}
\newcommand{\pflong}{\textit{Plasmodium falciparum}\xspace}
\newcommand{\pfshrt}{\mbox{\textit{P.\,falciparum}}\xspace}

%%%%BIBLIOGRAPHY

%Supplementary File Table Numbers:
\newcommand{\expdata}{S1\xspace}
\newcommand{\seqdata}{S2\xspace}
\newcommand{\protdata}{S3\xspace}
\newcommand{\blast}{S4\xspace}
\newcommand{\tabvar}{S5\xspace}
%Supplementary File Fig Numbers:

\newcommand{\expcor}{S1\xspace}
\newcommand{\expdistATCC}{S3\xspace}
\newcommand{\specialcell}[2][c]{%
	\begin{tabular}[#1]{@{}c@{}}#2\end{tabular}}
\newcommand{\beginsupplement}{%
	\setcounter{table}{0}
	\renewcommand{\thetable}{S\arabic{table}}%    %thetable references table counter 
	\setcounter{figure}{0}
	\renewcommand{\thefigure}{S\arabic{figure}}%
	\setcounter{equation}{0}
	\renewcommand{\theequation}{S\arabic{equation}}%

}
\renewcommand{\thesection}{}
\renewcommand{\thesubsection}{}
\renewcommand{\thesubsubsection}{}
\usepackage{setspace}
%adjust spacing
\doublespacing
\usepackage{titlesec}

\titlespacing\section{0pt}{12pt plus 2pt minus 2pt}{0pt plus 1pt minus 1pt}
\titlespacing\subsection{0pt}{12pt plus 2pt minus 2pt}{0pt plus 1pt minus 1pt}
\titlespacing\subsubsection{0pt}{12pt plus 2pt minus 2pt}{0pt plus 1pt minus 1pt}

% below 3 lines will put ALL table captions at the top...not sure if
% this is what we want but it is good enough for now

% \usepackage{float}
\floatstyle{plaintop}
\restylefloat{table}
%%%%%%%%%%%%%%%%%%%%%%%%%%%%%%%%%%%%%%%%%%%%%%%%%%%%%%%%%%%%%%%%%%%
        
        \definecolor{atomictangerine}{rgb}{1.0, 0.6, 0.4}
        \definecolor{darkbrown}{rgb}{1.0, 0.56, 0.24}
        \colorlet{darkcol}{black!30!white}
        \colorlet{lightcol}{black!10!white}
        \definecolor{txtcol}{HTML}{F40000}


%%%%%%%%%%%%%%%%%%%%%%%%%%%%%%%%%%%%%%%%%%%%%%%%%%%%%%%%%%%%%%%%%%%
\begin{document}

\widowpenalty10000
\clubpenalty10000

%\linenumbers %for line numbers
\onecolumn
%\twocolumn[  
%       \begin{@twocolumnfalse}
%               \begin{center}
                        \maketitle
%               \end{center}
%                       \bigskip

\thispagestyle{empty}
\noindent \textsuperscript{1} Department of Biology, McMaster University, Hamilton, ON, Canada

\newpage
\tableofcontents
\newpage

\section{Abstract}

\newpage
       

\section{Literature Review/Proposal}

\subsection{Introduction - Maybe move to Abstract, it encompasses everything}
After many years of trying to adapt to educational innovations of the 21st Century, educators will soon be looking into the not too distant future and considering what modern learning in post-secondary education will look like in the 22nd Century. Technological advances will continue to transform the landscape of the classroom and the use of digital games, although a source of controversy and debate, will prove to be highly effective instructional tools that can promote problem solving skills and critical thinking. Their use has already started to change the structure of the class in the elementary classroom as they serve to energize and engage students in their learning and improve learning, creativity and imagination. The traditional construct of the classroom has started to evolve, embracing a new way in which knowledge, skills and attitudes can be acquired. Cells at War is a collection of scientific video games developed through interdisciplinary efforts that reinforces the notion that video games are highly effective tools used to enhance instructional methods. (Not only enhancing but teaching and engaging, ask rosa). – I don’t know if I should mention the game but I like this line.

\subsection{Children and Play}

Although technological ecosystems have been established in schools for some time, its landscape continues to be missing significant components to create a fully functioning ecosystem. One of these key components is that educators are reticent to embrace digital technology and gaming as the new norm in the classroom. Kindergarten classrooms in Ontario are bustling places of activity, reflective of active inquiry through play. Exploration and provocations drive their curiosity while simultaneously developing their oral language and critical thinking skills \citep{kindergartencurriculum}. Digital games on ipads are being used to develop literacy and numeracy skills. Some educators however, oppose the use of digitized games especially in higher education and have not fully embraced the notion that learning can actually take place through play. We believe that play is paramount for children in their formative years and educational theorists like Vygotsky affirm this. He was firm in his belief that play is the leading line of development during preschool years \citep{vygotsky1967play}. Unfortunately, in many classrooms, educators continue to feel more comfortable in engaging in antiquated and traditional methods of instruction that consist of a didactic approach to learning. The need to challenge current educational practices is paramount as educators are currently teaching students from the perspective of their own past rather than the realities of the students’ future. – I like this line, its powerful that’s why I highlighted it.

\subsection{What Happens to Play Beyond the Formative Years?}

The need to engage and motivate learners does not cease to be a need in a student’s educational career. However, what does become a priority in higher academic years are measurable student outcomes \citep{ball2012politics,shore2010beyond,leather2021pedagogy}. Tests and quizzes take precedence over the formative process of learning. Assessment of learning (tests and quizzes) dominates an educator’s instructional practices which in turn diminishes the value of the experiential learning process (assessment as learning). Assessment methods seem to be based on antiquated teaching practices, which have been influenced by our Victorian educational heritage that is rooted in a persistent view of fear of play \citep{wood2013play}. This rigidity in instructional norms appears to have taken hold in our classrooms which diminishes the value and importance of games in the classroom. \citet{reich2020failure} argues that schools are complex and conservative institutions that have little incentive to adopt or test novel approaches. 

It can be argued that this fear of play is also due in large part to limited resources and training being provided to educators. Teachers require technological and pedagogical supports to develop their understanding of game based learning in the classroom. The expectation therefore to incorporate new and innovative teaching methodologies in the classroom is unrealistic without providing teachers with the appropriate training needed to do so. Incorporating game based learning in teacher practice is impacted by a lack of professional development opportunities that can systematically guide them in using games for teaching, learning, and assessment \citep{fishman2014empowering,ruggiero2013video,foster2020principles}. Might want to add a line here, don’t wanna end on quote


\subsection{Do Games Help or Hurt Learning: The Debate Over Game-Based Learning}

The debate of whether game based learning helps or hinders learning is ongoing. What comes into question in opposition to game based learning is that it serves to act as a source of entertainment and amusement rather than for educative purposes. Many individuals associate play with children, and therefore believe that play is trivial and unimportant (Play and learning Prensky). In fact, one does not learn to play, but rather plays to learn. Play is where a spark is ignited and the source in which curiosity is peaked. It is not silly, trivial or irrelevant but has a biological and evolutionary function which has a strong connection to learning (Play and Learning Presnky). It is, as Danny Hillis, founder of Thinking Machines and a former Disney Fellow has stated something every single culture does (play and learning Prensky).

Arguing the merits behind the development and creation of digital games for use in post-secondary classrooms can appear effortless. However, what cannot be overlooked at times is that educators spend a great deal of time planning and executing engaging lessons which can become time consuming and onerous. Svinicky and McKeachie (2013) argue that the biggest barrier to the use of games therefore is logistic. Although games can be fun and engaging, we know that our classrooms today are comprised of diverse students with varying needs and a one size fits all approach does not work.

\subsection{The Benefits of Using Video Games at the Post-secondary Level}

Classrooms are becoming a place where the student is now involved in the learning process. Rules, goals and objectives, outcomes, competition, interaction, and story are what Mark Prensky refers to as the six key structural elements of games (Prensky digital game based learning). These powerful factors are what set games apart from other engaging activities. Traditional forms of educational media lack what video games provide, a high degree of interactivity (coller, scott, mechanical engineering paper). Video games challenge individuals to work within specific parameters and respond to situations that occur in the game (coller, scott, mechanical engineering). The way in which individuals respond to challenges presented in video games allows them to think critically, strategize, and work collaboratively in an interactive and engaging manner.

Video games are visually appealing, stimulating, and heighten our senses in many ways. We become addicted to the hormone boost our brains receive when we achieve results (Lubomir Tomaska). They allow us to learn through doing and creating (video games and education Kurt Squire). Like play, upon beginning a video game, one is completely immersed into a whole new world without boundaries (coller, scott, engineering). Each player is engaged in a creative process where they are free to discover the limitless possibilities contained within the game. In a school setting, we are often taught there is one solution or right answer or way of doing something. Video games challenge this by giving players multiple entry points and pathways to consider.

The growing evidence of the effectiveness of video games in the classroom is emerging, and the value of game based learning is evident throughout various disciplines. In 2005, Northern Illinois University began teaching an undergraduate course in Numerical Methods using a video game called “NIU-Torcs” (Coller, Scott). Students were observed spending twice the average amount of time on coursework than other subject areas (Coller, Scott). Other games are being developed for nursing education which focus on developing medication calculation skills. This is an area of global concern perhaps due to test anxiety and low mathematical self-efficacy in nursing students (Foss, Mordt Medication calculation).  Game based learning not only provides students with infinite possibilities for discovery but sustains student interest, builds confidence, and helps to improve cognition and learning outcomes.


\subsection{Concluding the Lit Review - I dont know if I wanna put this here}

The long-term positive effects of Game-Based Learning will continue to be researched and studied. Although empirical evidence is slowly emerging but still limited, studies and research to date are proving that Game-Based Learning has the potential to transform post secondary classrooms. They are moving from teacher centred to student centred learning hubs which serve to create a learning environment in which students are able to take risks, explore, discover, fail and try again.

\clearpage\newpage
\section{References}

%%%FIGURES%%%%%

%%%%PRINTING BIBLIOGRAPHY%%%%
\nocite{*}
\printbibliography[heading=none, sorting=nyt]
\newpage



\end{document}



